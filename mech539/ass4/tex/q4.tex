\section{Airfoil comparison}
\subsection{Drag polar}
The drag polars are shown in~\Cref{fig:q4_polars}. Typically, a higher lift-to-drag (L/D) ratio
is one of the most imperative goals in airfoil design.
Since constant L/D lines have also been plotted, the comparison becomes straight-forward:
\begin{itemize}
    \item The NACA 0012 airfoil has a lower L/D ratio than the other two airfoils
        for any value of lift coefficient.
    \item Except at low lift coefficients, the DAE31 airfoil is the best airfoil in terms
        of L/D ratio.
\end{itemize}
The other factor that may be considered is the following: it is desirable for an airfoil
to attain high lift at relatively low angles of attack. This is especially true when
doing simulations with a potential flow solver, since it cannot predict stall although
we know it occurs for all airfoils at some point. Comparing airfoils this way basically
leads to the same ranking as above.
\begin{figure}
    \centering
    \includegraphics[width=\textwidth]{./figs/q4_polars}
    \caption{Drag polars for all three airfoils. Drag at $C_L = 1$ is shown as
        empty circle.}\label{fig:q4_polars}
\end{figure}

\subsection{Drag at fixed lift}
The drag at lift coefficient equal to 1.0 for all three airfoils is tabulated in
~\Cref{tab:q4}. It has been interpolated using the data shown in the drag polars
plot. The table shows that the DAE31 airfoil exhibits the lowest drag coefficient
at that particular lift coefficient. GA(W) is the second best and NACA0012 is the
worst out of the three.

For the sake of completeness, it should be noted that low drag coefficients are usually
desirable when designing airfoils for aerodynamics.
\begin{table}
    \centering
    \caption{Drag coefficients at equivalent lift coefficient of 1.0}
    \label{tab:q4}
    \input{tableq4}
\end{table}

\subsection{Pressure distribution}
The pressure distribution for all airfoils is shown on~\Cref{fig:q4_cp}. The following
observations can be made:
\begin{itemize}
    \item The NACA0012 and GA airfoils produce most of their lift close to the leading edge;
        that is where the difference in pressure between top and bottom surface is greatest.
        In fact, compared to DAE31, GA and NACA exhibit large peaks in pressure
        right at the leading edge, after which the pressure quickly increases on the top surface
        and decreases on the bottom surface.
    \item GA produces more lift near the trailing edge than NACA, which produces
        low lift after around 80\% of the length.
    \item The DAE31 airfoil produces most of its lift over the first 60\%. The pressure
        on the bottom surface remains relatively constant over most of the length, and the
        pressure on the top surface decreases way more gradually than for the other two
        airfoils.
\end{itemize}

\begin{figure}
    \centering
    \includegraphics[width=\textwidth]{./figs/q4_cp}
    \caption{Pressure distribution for all three airfoils at angle of attack of
        4 degrees.}\label{fig:q4_cp}
\end{figure}

\subsection{Skin friction}
Skin friction plots are shown in~\Cref{fig:q4_cf}. Transition locations are also shown.
It can be seen that, on the upper surface, the flow transitions at around 15\%
for both NACA and GA while the transition occurs at 37\% for DAE. The effect
of transition on skin friction coefficient is easy to see: transition leads to a sharp increase
in skin friction coefficient. In fact, that is how it was detected. This can physically
be explained by the fact that turbulent flow has a larger velocity gradient in the normal
direction, mathematically written as $\partial u/\partial y$ in this case, than laminar flow.
The shear stress at the wall $\tau_w$ is given by:
$$
    \tau_w = \mu \left(\frac{\partial u}{\partial y}\right)_{y=0}
$$
and thus increases linearly with $\partial u/\partial y$. The skin friction coefficient
is nothing but a non-dimensionalized $\tau_w$.

The transition location can also be linked to the pressure distribution plots in the previous
section as follows:
transition leads to a sharp increase in pressure. In other words, the earlier the transition
the earlier the pressure tends to increase back up to freestream conditions. Of course, this
matches the results shown in~\Cref{fig:q4_cp}, meaning that $C_P$ for DAE stays relatively
high much longer than for GA and NACA.

The same can be said for the lower surface, except that in case pressure decreases back to
freestream value when transition occurs. Interestingly, there is no transition on the lower surface
for DAE! Consequently, the lower surface pressure stays relatively constant over most of the
airfoil's length.

Finally, the following conclusion can be made: delaying the transition as much
as possible leads to better overall aerodynamic performance.
\begin{figure}[H]
    \centering
    \includegraphics[width=\textwidth]{./figs/q4_cf}
    \caption{Skin friction coefficient for all airfoils at an angle of attack of
        4 degrees.}\label{fig:q4_cf}
\end{figure}

