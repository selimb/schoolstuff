\section{Aerodynamic characteristics at various angles}
\subsection{Pressure coefficient}
The $c_P$ distributions are shown in~\Cref{fig:q3_cp}. It can be seen that the difference
in pressure between the top and bottom surface increases as the angle of attack increases. Incrementing
the angle of attack has the effect of decreasing $c_P$ everywhere on the top surface and increasing
$c_P$ everywhere on the bottom surface.

One could also look at it in terms of velocity: increasing the angle of attack also has a positive
effect on the acceleration of the fluid at the leading edge. In other words, the flow turns more
more agressively. Conversely, the velocity decreases as the angle of attack increases.

\begin{figure}
    \centering
    \includegraphics[width=0.8\textwidth]{./figs/q3_cp}
    \caption{Effect of angle of attack on pressure distribution. The angles of attack vary from 0 to
        10 in increments of 2.}
    \label{fig:q3_cp}
\end{figure}

\subsection{Lift and drag}
