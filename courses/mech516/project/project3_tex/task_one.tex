%!TEX root = main.tex

\section{Numerical experiments with a TVD scheme for the Euler equations.}

While it is suggested/required that the $\Delta t$ is kept constant, I found it was easier and more fail-proof to use a technique similar to the one in the last mini-project, where the Courant number and $\Delta x$, along with the maximum eigenvalues, fully constrain $\Delta t$. Of course, it was ensured that the final time step was at 25 with a simple check.

The shock tube problem is illustrated in~\Cref{fig:riemann}.
\begin{figure}[H]
    \centering
    \includegraphics[width=0.8\textwidth]{figs/riemann}
    \caption{Task 1 Wave Propagation}\label{fig:riemann}
\end{figure}

\subsection{Comparison of solutions obtained with and without a slope limiter.}
\begin{quote}
 \em \centering The Courant number was set to 0.8.
\end{quote}
Results are shown in~\Cref{fig:ex1_1} -- \emph{Minmod} was chosen as the extra limiter.

\begin{figure}[H]
    \centering
    \includegraphics[width=1.0\textwidth]{figs/ex1_1}
    \caption{Exercise 1.1 Results}\label{fig:ex1_1}
\end{figure}

The following observations can be made about the usage of each limiter:
\begin{description}
    \item [Zero:] Using a zero limiter essentially sets $\widetilde{U}_i = U_i^n$ and results in the scheme behaving \emph{exactly} like Godunov's scheme, i.e. the results are identical, albeit probably a bit slower. Thus, using a zero limiter yields a first-order upwind scheme!
    \item [Average:] As mentioned in the notes, slope limiters aim to provide second-order accuracy. However, these second-order schemes exhibit oscillations near sharp gradients. In this case, dispersive errors clearly dominate. Thus, the scheme behaves like a second-order upwind scheme when an average slope is used. The average slope method is then a perfect example of a lack of limiter! It goes without saying that the TVD property is not achieved in this case.
    \item [Minmod:] As expected, the Minmod slope limiter shows the highest accuracy, compared to the above two. The scheme also doesn't oscillate like when using average slope. TVD property is achieved.
\end{description}

\subsection{Study of different limiters.}
\begin{quote}
 \em \centering The Courant number was set to 0.1 here because it amplifies the difference between limiters (mostly for graphical purposes).
\end{quote}
\Cref{fig:ex1_2} shows the results -- VanLeer (smooth) was the chosen limiter. From \texttt{Limiters.pdf} on MyCourses, it is directly possible to see where the limiters lie on a Sweby diagram. It is also known that Superbee and Minmod are the most and least compressive limiters, respectively. In other words, the Superbee limiter applies the minimum possible limiting, while remaining in the TVD region, while Minmod does the opposite. Consequently, dissipation is more dominant in Minmod than in Superbee, where the latter is known to suffer from excessive sharpening. In fact, the Superbee curve follows the Lax-Wendroff curve for $r > 1$. The Superbee limiter can then be thought of as more \emph{agressive} than the Minmod limiter.

VanLeer, from the Sweby diagram, is essentially a compromise between the two aforementioned limiters.

\Cref{tab:task1} tabulates relative computational cost and accuracy for each limiter. This is done exactly as in \href{https://github.com/Kreger51/mech_516/blob/master/project/docs/MECH516_Project2_260450544.pdf}{Mini-Project 2}.\\[-3mm]
\noindent\hrulefill\par
\noindent\makebox[\textwidth][c]{%
    \begin{minipage}{0.8\textwidth}
    \hrulefill\par
    \begin{table}[H]
        \ra{0.8}
        \caption{Relative Accuracy and Cost for three chosen limiters. Error uses averaged $L_2$ norm and is averaged across primitive variables.}
        \label{tab:task1}
        \centering
        \begin{tabular}{ccc}
            \toprule
            Limiter & Error & Cost\\
            \midrule
            Minmod & 1 & 1\\
            VanLeer & 0.85 & 0.98\\
            Superbee & 0.7 & 0.99\\
            \bottomrule
        \end{tabular}
    \end{table}
    \end{minipage}%
}\\[1cm]
It can then be seen that, while computational cost is virtually the same for each limiter, the Superbee limiter leads to more accurate results -- 30\% error reduction compared to Minmod. VanLeer's accuracy is exactly in between Minmod and Superbee, as predicted.
\begin{figure}
    \centering
    \includegraphics[width=1.0\textwidth]{figs/ex1_2}
    \caption{Exercise 1.2 Results}\label{fig:ex1_2}
\end{figure}
