\section{Pressure distribution}
The pressure distribution is shown in~\Cref{fig:cp}.
The following observations can be made:
\begin{enumerate}
    \item As expected from an airfoil with an angle of attack higher than zero,
        the pressure is lower on the upper surface than on the lower surface over most of
        the airfoil length: this should lead to lift! In other words, the flow is
        accelerated, or turned, over the upper surface.
    \item On the upper surface, the pressure initially drops rather quickly and then
        slowly increases over the length of the airfoil. Consequently, the flow is
        going through an \textit{adverse} pressure gradient.
    \item On the lower surface, the pressure increases until the stagnation point,
        after which it decreases -- favorable pressure gradient -- up to around 70\%
        of the length. After this point, the pressure gradient comes adverse.
    \item Over approximately the last 10\%, the coefficient of pressure on the upper surface
        is \textit{positive}. Moreover, pressure on the upper surface increases past that
        of the lower surface -- this decreases lift.
\end{enumerate}

\begin{figure}
    \centering
    \includegraphics[width=\textwidth]{./figs/cp.pdf}
    \caption{Coefficient of pressure distribution over the airfoil surface.}\label{fig:cp}
\end{figure}


