\documentclass{SelimArticle}
%!TEX root = main.tex

%%%%%%%%%%%%%%%%%%%%%%%%%%%%%%%
% Additional Packages/Options %
%%%%%%%%%%%%%%%%%%%%%%%%%%%%%%%
% \setlist{nosep}
\hypersetup{hidelinks}

%%%%%%%%%%%%%%%%%
% Title Options %
%%%%%%%%%%%%%%%%%
\course{Computational Gas Dynamics}
\coursenum{MECH 516} 
%Add \\[0.3cm] for other line. 
\title{Assignment 1}   
\student{Selim \textsc{Belhaouane}}  
\studentnum{260450544}
\date{\today}

%%%%%%%%%%%%%%%%%%%%%%%%%   
% Additional Formatting %
%%%%%%%%%%%%%%%%%%%%%%%%%
%Horizontal line below section.
\sectionfont{ \sectionrule{0pt}{0pt}{-5pt}{0.8pt} }  
\setcounter{secnumdepth}{3}  
\setcounter{tocdepth}{2} 
\numberwithin{equation}{section}
\numberwithin{figure}{section}

\newcommand{\ra}[1]{\renewcommand{\arraystretch}{#1}}
\begin{document}
\mytitlepage
% \tableofcontents
\newpage
% Begin writing here.
\begin{abstract}\em
Code is hosted on \href{https://github.com/Kreger51/mech_516/tree/master/project}{GitHub}~\cite{code}. The Euler equation schemes are in \texttt{euler/schemes.py}, the linear schemes are in \texttt{linear/schemes.py}, \texttt{linear/solvers.py} and the Riemann solver is in \texttt{riemann/*}. Because of time constraints, this is far from the cleanest and most efficient code I've ever written. You can view how the code is used and implemented by viewing the \href{http://nbviewer.ipython.org/github/Kreger51/mech_516/tree/master/project/}{IPython Notebooks}~\cite{notebook}.
\end{abstract}
%!TEX root = main.tex

\section{Numerical experiments with a TVD scheme for the Euler equations.}

While it is suggested/required that the $\Delta t$ is kept constant, I found it was easier and more fail-proof to use a technique similar to the one in the last mini-project, where the Courant number and $\Delta x$, along with the maximum eigenvalues, fully constrain $\Delta t$. Of course, it was ensured that the final time step was at 25 with a simple check.

The shock tube problem is illustrated in~\Cref{fig:riemann}.
\begin{figure}[H]
    \centering
    \includegraphics[width=0.8\textwidth]{figs/riemann}
    \caption{Task 1 Wave Propagation}\label{fig:riemann}
\end{figure}

\subsection{Comparison of solutions obtained with and without a slope limiter.}
\begin{quote}
 \em \centering The Courant number was set to 0.8.
\end{quote}
Results are shown in~\Cref{fig:ex1_1} -- \emph{Minmod} was chosen as the extra limiter.

\begin{figure}[H]
    \centering
    \includegraphics[width=1.0\textwidth]{figs/ex1_1}
    \caption{Exercise 1.1 Results}\label{fig:ex1_1}
\end{figure}

The following observations can be made about the usage of each limiter:
\begin{description}
    \item [Zero:] Using a zero limiter essentially sets $\widetilde{U}_i = U_i^n$ and results in the scheme behaving \emph{exactly} like Godunov's scheme, i.e. the results are identical, albeit probably a bit slower. Thus, using a zero limiter yields a first-order upwind scheme!
    \item [Average:] As mentioned in the notes, slope limiters aim to provide second-order accuracy. However, these second-order schemes exhibit oscillations near sharp gradients. In this case, dispersive errors clearly dominate. Thus, the scheme behaves like a second-order upwind scheme when an average slope is used. The average slope method is then a perfect example of a lack of limiter! It goes without saying that the TVD property is not achieved in this case.
    \item [Minmod:] As expected, the Minmod slope limiter shows the highest accuracy, compared to the above two. The scheme also doesn't oscillate like when using average slope. TVD property is achieved.
\end{description}

\subsection{Study of different limiters.}
\begin{quote}
 \em \centering The Courant number was set to 0.1 here because it amplifies the difference between limiters (mostly for graphical purposes).
\end{quote}
\Cref{fig:ex1_2} shows the results -- VanLeer (smooth) was the chosen limiter. From \texttt{Limiters.pdf} on MyCourses, it is directly possible to see where the limiters lie on a Sweby diagram. It is also known that Superbee and Minmod are the most and least compressive limiters, respectively. In other words, the Superbee limiter applies the minimum possible limiting, while remaining in the TVD region, while Minmod does the opposite. Consequently, dissipation is more dominant in Minmod than in Superbee, where the latter is known to suffer from excessive sharpening. In fact, the Superbee curve follows the Lax-Wendroff curve for $r > 1$. The Superbee limiter can then be thought of as more \emph{agressive} than the Minmod limiter.

VanLeer, from the Sweby diagram, is essentially a compromise between the two aforementioned limiters.

\Cref{tab:task1} tabulates relative computational cost and accuracy for each limiter. This is done exactly as in \href{https://github.com/Kreger51/mech_516/blob/master/project/docs/MECH516_Project2_260450544.pdf}{Mini-Project 2}.\\[-3mm]
\noindent\hrulefill\par
\noindent\makebox[\textwidth][c]{%
    \begin{minipage}{0.8\textwidth}
    \hrulefill\par
    \begin{table}[H]
        \ra{0.8}
        \caption{Relative Accuracy and Cost for three chosen limiters. Error uses averaged $L_2$ norm and is averaged across primitive variables.}
        \label{tab:task1}
        \centering
        \begin{tabular}{ccc}
            \toprule
            Limiter & Error & Cost\\
            \midrule
            Minmod & 1 & 1\\
            VanLeer & 0.85 & 0.98\\
            Superbee & 0.7 & 0.99\\
            \bottomrule
        \end{tabular}
    \end{table}
    \end{minipage}%
}\\[1cm]
It can then be seen that, while computational cost is virtually the same for each limiter, the Superbee limiter leads to more accurate results -- 30\% error reduction compared to Minmod. VanLeer's accuracy is exactly in between Minmod and Superbee, as predicted.
\begin{figure}
    \centering
    \includegraphics[width=1.0\textwidth]{figs/ex1_2}
    \caption{Exercise 1.2 Results}\label{fig:ex1_2}
\end{figure}

%!TEX root = main.tex
\section{Numerical experiments with the Euler equations.}
\begin{quote}
\em The chosen schemes are Godunov, Roe and McCormack.
\end{quote}
As per the instructions, $\Delta x$ = 1. The Courant number was chosen to be constant 0.9 for \emph{most} of the simulation, so as to satisfy the CFL condition. The Courant number for Euler equations is given in~\Cref{eq:courant_euler}.
\begin{equation}
    \label{eq:courant_euler}
    \nu = \dfrac{\Delta t}{\Delta x}\cdot\mathrm{max}(|\lambda|)
\end{equation}
where $\mathrm{max}(|\lambda|)$ is the absolute maximum eigenvalue across all nodes. Since this value may change between iterations, the time step needs to be recalculated for every iteration in time:
\begin{equation}
    \Delta t = \dfrac{\nu\cdot\Delta x}{\mathrm{max}(|\lambda|)}
\end{equation}
However, results are requested for $t = 25$ exactly. Thus, the Courant number may be ``un-fixed'' in order to achieve a time step that satisfies the final $t$ requirement. The Courant numbers resulting from this were 0.89, 0.12 and 0.87 for the Godunov, Roe and McCormack simulations, respectively. It is good that they satisfy the CFL condition.

Density, velocity and pressure plots can be found in~\Cref{app:2}. The Riemann problem is illustrated in~\Cref{fig:riemann}.
\begin{figure}[H]
    \centering
    \includegraphics[width=0.9\textwidth]{./figs/riemann}
    \caption{Task 2 Riemann Representation.}\label{fig:riemann}
\end{figure}
\Cref{tab:task2} tabulates computational cost and accuracy for each scheme. Cost was measured in milliseconds taken to complete the simulation. The accuracy was measured by calculating the errors for each primitive between a given scheme and the exact solution with a method similar to the one in Task 1, and averaging the three returned errors. In other words:
\begin{align*}
    e &= \dfrac{e_\rho + e_u + e_p}{3}
\end{align*}
where $e_\rho$, $e_u$, $e_p$ are the errors for density, velocity and pressure respectively and are calculated as in Task 1.\\[-3mm]
\noindent\hrulefill\par
\noindent\makebox[\textwidth][c]{%
    \begin{minipage}{0.8\textwidth}
    \hrulefill\par
    \begin{table}[H]
        \ra{0.8}
        \caption{Relative Accuracy and Cost for three chosen schemes. Error uses averaged $L_2$ norm and is averaged across primitive variables.}
        \label{tab:task2}
        \centering
        \begin{tabular}{ccc}
            \toprule
            Scheme & Error & Cost\\
            \midrule
            Godunov & 1 & 1\\
            Roe & 1 & 0.02\\
            McCormack & 0.77 & 0.23\\
            \bottomrule
        \end{tabular}
    \end{table}
    \end{minipage}%
}\\[1cm]
\noindent From~\Cref{tab:task2}, the following can be said:
\begin{description}
    \item [First Order:]\hfill \\
    Godunov and Roe have similar accuracy. However, the computation cost of Godunov's scheme is \emph{huge} -- Roe's scheme takes 2\% of the time. This gap could be further increased if the program was parallelized -- Godunov's scheme performs very poorly because of all the conditionals. As a side note, programmer time of Godunov's scheme is also much larger, since an exact Riemann solver needs to be implemented -- Roe's scheme is fairly straightforward.
    \item [Godunov vs. McCormack:]\hfill\\
    While the gap in computation cost is not as large between these two schemes, the second order scheme is certainly more accurate -- the error is reduced by 23\%.
    \item [Roe vs. McCormack:]\hfill\\
    Roe's scheme is about 10 times as fast as McCormack's, however the latter is more accurate.
    \item [First Order vs. Second Order:]\hfill\\
    As in Task 1, it can be seen that second order schemes are dispersive and that first order schemes are dissipative. The former provides more accuracy in both cases. Thus, if one were to switch from Godunov to either a first-order or second-order scheme, the question to ask would be: ``Do you want low computation time or accuracy?''
\end{description}

\bibliographystyle{IEEEtran}
\bibliography{dabib}
\newpage
\newgeometry{margin=1.2cm}
\pagestyle{empty}
\appendix
\titleformat{\section}[block]{\normalfont\large\bfseries}{\thesection.}{0.5em}{}
\titleformat{\subsection}[block]{\normalfont\large}{\thesubsection.}{0.5em}{}
%!TEX root = main.tex
\section{Task 1}
\subsection{Exercise 1.1}
\label{app:11}
\begin{figure}[H]
    \centering
    \includegraphics[width=0.9\textwidth]{./figs/1-1_courant-08_5}
    \caption{$\nu = 0.8$ - Five first time steps}% \label{fig:1-1_courant-08_5}
\end{figure}
\begin{figure}[H]
    \centering
    \includegraphics[width=0.7\textwidth]{./figs/1-1_courant-08}
    \caption{$\nu = 0.8$ - Time = 100}% \label{fig:1-1_courant-08}
\end{figure}
\begin{figure}[H]
    \centering
    \includegraphics[width=0.9\textwidth]{./figs/1-1_courant-102_5}
    \caption{$\nu = 1.02$ - Five first time steps}% \label{fig:1-1_courant-102_5}
\end{figure}
\begin{figure}[H]
    \centering
    \includegraphics[width=0.7\textwidth]{./figs/1-1_courant-102}
    \caption{$\nu = 1.02$ - Time = 100}% \label{fig:1-1_courant-102}
\end{figure}
\begin{figure}[H]
    \centering
    \includegraphics[width=0.9\textwidth]{./figs/1-1_courant-2_5}
    \caption{$\nu = 2.0$ - Five first time steps}% \label{fig:1-1_courant-2_5}
\end{figure}
\begin{figure}[H]
    \centering
    \includegraphics[width=0.7\textwidth]{./figs/1-1_courant-2}
    \caption{$\nu = 2.0$ - Time = 100}% \label{fig:1-1_courant-2}
\end{figure}

% \newpage
%%%%%%%%%%%%%%%%%%
%% Exercise 1.2 %%
%%%%%%%%%%%%%%%%%%
\subsection{Exercise 1.2}
All plots below show the numerical solution at $t = 100$ for both schemes and signals. \\
$\nu = 0.9$.
\label{app:12}
\begin{figure}[H]
    \centering
    \includegraphics[width=0.87\textwidth]{./figs/1-2_LF_triangle}
    \caption{Lax-Friedrichs - Triangular Signal}\label{fig:1-2_LF_triangle}
\end{figure}
\begin{figure}[H]
    \centering
    \includegraphics[width=0.87\textwidth]{./figs/1-2_LF_smooth}
    \caption{Lax-Friedrichs - Smooth Signal}\label{fig:1-2_LF_smooth}
\end{figure}
\begin{figure}[H]
    \centering
    \includegraphics[width=0.87\textwidth]{./figs/1-2_LW_triangle}
    \caption{Lax-Wendroff - Triangular Signal}\label{fig:1-2_LW_triangle}
\end{figure}
\begin{figure}[H]
    \centering
    \includegraphics[width=0.87\textwidth]{./figs/1-2_LW_smooth}
    \caption{Lax-Wendroff - Smooth Signal}\label{fig:1-2_LW_smooth}
\end{figure}

%%%%%%%%%%%%%%%%%%
%% Exercise 1.3 %%
%%%%%%%%%%%%%%%%%%
\subsection{Exercise 1.3}
\label{app:13}
\begin{figure}[H]
    \centering
    \includegraphics[width=0.8\textwidth]{./figs/1-3_LF_triangle}
    \caption{Lax-Friedrichs - Triangular Signal}\label{fig:1-3_LF_triangle}
\end{figure}
\begin{figure}[H]
    \centering
    \includegraphics[width=0.8\textwidth]{./figs/1-3_LF_smooth}
    \caption{Lax-Friedrichs - Smooth Signal}\label{fig:1-3_LF_smooth}
\end{figure}
\begin{figure}[H]
    \centering
    \includegraphics[width=0.8\textwidth]{./figs/1-3_LW_triangle}
    \caption{Lax-Wendroff - Triangular Signal}\label{fig:1-3_LW_triangle}
\end{figure}
\begin{figure}[H]
    \centering
    \includegraphics[width=0.8\textwidth]{./figs/1-3_LW_smooth}
    \caption{Lax-Wendroff - Smooth Signal}\label{fig:1-3_LW_smooth}
\end{figure}





%!TEX root = main.tex
\section{Task 2}
\label{app:2}
\begin{figure}[H]
    \centering
    \includegraphics[width=0.90\textwidth]{figs/godunov}
    \caption{Godunov results}\label{fig:godunov}
\end{figure}
\begin{figure}[H]
    \centering
    \includegraphics[width=0.90\textwidth]{./figs/roe}
    \caption{Roe results}\label{fig:roe}
\end{figure}
\begin{figure}[H]
    \centering
    \includegraphics[width=0.90\textwidth]{./figs/mccormack}
    \caption{McCormack results}\label{fig:mccormack}
\end{figure}


\end{document}