\documentclass{SelimArticle}
%!TEX root = main.tex

%%%%%%%%%%%%%%%%%%%%%%%%%%%%%%%
% Additional Packages/Options %
%%%%%%%%%%%%%%%%%%%%%%%%%%%%%%%
% \setlist{nosep}
\hypersetup{hidelinks}

%%%%%%%%%%%%%%%%%
% Title Options %
%%%%%%%%%%%%%%%%%
\course{Computational Gas Dynamics}
\coursenum{MECH 516} 
%Add \\[0.3cm] for other line. 
\title{Assignment 1}   
\student{Selim \textsc{Belhaouane}}  
\studentnum{260450544}
\date{\today}

%%%%%%%%%%%%%%%%%%%%%%%%%   
% Additional Formatting %
%%%%%%%%%%%%%%%%%%%%%%%%%
%Horizontal line below section.
\sectionfont{ \sectionrule{0pt}{0pt}{-5pt}{0.8pt} }  
\setcounter{secnumdepth}{3}  
\setcounter{tocdepth}{2} 
\numberwithin{equation}{section}
\numberwithin{figure}{section}

\newcommand{\ra}[1]{\renewcommand{\arraystretch}{#1}}
\begin{document}
\maketitle
% Begin writing here.
\newcommand{\kw}{$k-\omega$}
\section{Dimensionless Variables}
\newcommand{\pinf}{\ensuremath{P_\infty}}
\newcommand{\rinf}{\ensuremath{\rho_\infty}}
\newcommand{\minf}{\ensuremath{\mu_\infty}}
\newcommand{\pri}{\ensuremath{\pinf/\rinf}}
\subsection{Current State}
The Navier-Stokes momentum equations in \texttt{Syn3D} are currently non-dimensionalized
as follows, where $\overline{\phi}$ and $\phi$ are dimensional and dimensionless, respectively:
\begin{gather*}
    \rho = \frac{\overline{\rho}}{\rinf}
    \qquad u = \frac{\overline{u}}{\sqrt{\pri}}
    \qquad P = \frac{\overline{P}}{\pinf}
    \\
    \mu_V = \frac{\overline{\mu_V}}{\minf}
    \qquad t = \frac{\overline{t}\cdot\sqrt{\pri}}{c}
    \qquad x = \frac{\overline{x}}{c}
\end{gather*}
where $c$ is the chord length, also often referred to as the reference length $L$.

\subsection{Turbulent Quantities}
The \kw equations also have to be non-dimensionalized and it has to be consistent with the
Navier-Stokes equations. Both the NASA Turbulence Modeling Resource and \texttt{CFL3D}
use the following:
\begin{gather*}
    \mu_T = \frac{\overline{\mu}_T}{\minf}
    \qquad k = \frac{\overline{k}}{a_\infty^2}
    \qquad \omega = \frac{\overline{\omega}\minf}{\rinf a_\infty^2}
    \qquad \Omega = \frac{\overline{\Omega}c}{a_\infty}
\end{gather*}
where $a$ is the speed of sound.

The above needs to be slightly tweaked to match \texttt{Syn3D}. In fact, replacing
occurences of $a$ with $\sqrt{\pri}$ is all that is necessary, since those two are related
by a factor of $\sqrt{\gamma}$. Consequently, we can expect the nondimensionalized equations
to be very similar, only varying by factors of $\sqrt{\gamma}$.

The following nondimensionalization is then used:
\begin{gather*}
    \mu_T = \frac{\overline{\mu}_T}{\minf}
    \qquad k = \frac{\overline{k}}{\pri}
    \qquad \omega = \frac{\overline{\omega}\minf}{\rinf\pri}
    \\
    d = \frac{\overline{d}}{c}
    \qquad \Omega = \frac{\overline{\Omega}c}{\sqrt{\pri}}
\end{gather*}

\section{Nondimensionalized Equations}
In order for the equations to take unitless quantities as inputs and
also return unitless quantities, the dimensionless variables defined above need to be
substituted into the governing equations. In other words, one cannot solve dimensional equations
using dimensionless variables. The derivation is not shown here.
\subsection{Navier-Stokes}
\newcommand{\diff}[2]{\ensuremath{
    \frac{\partial #1}{\partial #2}
}}
\newcommand{\cmu}{\ensuremath{
    \sqrt{\gamma}\frac{M_\infty}{\mathit{Re}}
}}
The equations are slightly simplified to remain concise. While this is not technically part
of the nondimensionalization, it should be noted that the Boussinesq approximation is used and
the terms associated with molecular diffusion and turbulent transport in the energy equation
are neglected (see NASA's website).
\begin{gather}
    \diff{\rho}{t} + \diff{(\rho u)}{x} = 0\\
    \diff{( \rho u )}{t} + \diff{(\rho u u)}{x} = -\diff{P}{x}
        + \cmu\diff{}{x}\left( (\mu_V + \mu_T) \diff{u}{x} \right)
\end{gather}
\begin{align}
\begin{split}
    \diff{(\rho E)}{t} + \diff{(\rho u E + uP)}{x} =&
        \cmu\diff{}{x}\left[ (\mu_V + \mu_T)u\diff{u}{x} \right ]
    \\
    &- \cmu\diff{}{x}\left[
        \left(\frac{\mu_V}{\mathit{Pr}} + \frac{\mu_T}{\mathit{Pr}_T}\right)
        \frac{\gamma}{\gamma - 1}\diff{(P/\rho)}{x}
    \right ]
\end{split}
\end{align}
The term $\cmu$ comes from the chosen nondimensionalization.

\subsection{K-$\omega$ SST}
Not only must one nondimensionalize the transport equations, but it is necessary to perform
the same for the boundary conditions and additional functions.
\subsubsection{Transport Equations}
\begin{align}
    \rho\diff{k}{t} + \rho u \diff{k}{x} =& \cmu \mu_T \Omega^2 -
        - \left(\cmu\right)^{-1}\beta^*\rho k\omega
        + \cmu\diff{}{x}\left[ \left(\mu_V + \sigma_k\mu_T\right)\diff{k}{x} \right]\\
    \begin{split}
    \rho\diff{\omega}{t} + \rho u \diff{\omega}{x} =& \cmu\rho\gamma\Omega^2
        - \left(\cmu\right)^{-1}\beta^*\rho\omega^2
        + \cmu\diff{}{x}\left[ \left(\mu_V + \sigma_k\mu_T\right)\diff{\omega}{x} \right]\\
        &+ \cmu 2(1 - F_1)\frac{\rho \sigma}{\omega}\diff{k}{x}\diff{\omega}{x}
    \end{split}
\end{align}
\subsubsection{Additional Functions}
The turbulent eddy viscosity is computed from:
\begin{equation}
    \mu_T = \min\left(
        \frac{\rho k}{\omega},
        \left(\cmu\right)^{-1}\frac{\rho a_1 k}{\Omega F_2}
    \right)
\end{equation}
Blending functions are given by:
\begin{align*}
    F_1 = \tanh(arg_1^4)\quad &; \quad arg_1
        = \min\left[\max(\Gamma_1, \Gamma_2), \Gamma_3 \right]
    \\
    F_2 = \tanh(arg_2^2)\quad &; \quad arg_2 = \max\left(2\Gamma_1, \Gamma_1\right)
\end{align*}
where
\begin{align}
    \Gamma_1 &= \cmu \frac{\sqrt{k}}{\omega d}\\
    \Gamma_2 &= \left(\cmu\right)^2 \frac{500\mu_V}{\rho d^2 \omega}\\
    \Gamma_3 &= \frac{\pinf}{c^2}\frac{4\rho k}{\text{CD} d^2}\\
    \text{CD}       &= \max\left(
        \frac{\pinf}{c^2}2\rho\sigma\frac{1}{\omega}\diff{k}{x}\diff{\omega}{x},
        10^{-20}
    \right)
\end{align}
\textbf{The factors in $\text{CD}$ and $\Gamma_3$ are neglected since the solver sets
\pinf and $c$ to 1.  A note of this is also present in the code.}
\subsubsection{Boundary Conditions}
Freestream BCs for the flat plate case are
\begin{align}
    k_{farfield} &= 9\cdot10^{-9}\\
    \omega_{farfield} &= 10^{-6}
\end{align}
Viscous wall BCs are given as:
\begin{align}
    \omega_{wall} &= \left(\cmu\right)^2 60\frac{\mu_V}{\beta_1\rho d^2}\\
    k_{wall} &= 0
\end{align}

\subsubsection{Implementation}
The implementation is straight forward. However, it is relevant to note that both
$\mu_V$ and $\mu_T$ are injected with $\cmu$. This was originally done with $\mu_V$ in
\texttt{viscf.f}.

In other words, one can let:
\begin{align}
    \hat{\mu}_V &= \cmu\mu_V\\
    \hat{\mu}_T &= \cmu\mu_T
\end{align}
These can be substituted in the nondimensionalized equations/functions/BCs above.

\subsection{Spalart-Allmaras}
\newcommand{\sa}{\ensuremath{\hat{\nu}}}
The Spalart-Allmaras model solves for a variable \sa ~ related to the eddy viscosity
through
\begin{equation}
    \mu_T = \rho\sa f_{v_1}
\end{equation}
where
\begin{align}
    f_{v_1} &= \frac{\chi^3}{\chi^3 + C_{v_1}^3}\\
    \chi &= \frac{\sa}{\nu}
\end{align}
\subsubsection{Transport Equation}
\begin{align}
\begin{split}
    \diff{\sa}{t} + u\diff{\sa}{x} =~ &C_{b_1}(1 - f_{t_2})\Omega\sa\\
        &+ \cmu\left\{
            C_{b_1}\left[(1 - f_{t_2})f_{v_2} + f_{t_2}\right]\frac{1}{\kappa^2} - C_{w_1}f_w
        \right\} \left(\frac{\sa}{d}\right)^2\\
        &+ \cmu \frac{1}{\sigma}\diff{}{x}\left[
            \left(\sa + (1 + C_{b_2})\sa\right)\diff{\sa}{x}
        \right]\\
        &+ \cmu \frac{C_{b_2}}{\sigma}\left(\diff{\sa}{x}\right)^2
\end{split}
\end{align}
The terms are grouped slightly differently than in the original reference because of the common
factors. This is done in CFL3D.
\subsubsection{Additional Functions}
Only the functions that end up with an additional factor are shown.
\begin{align}
\begin{split}
    r &= \min(r_1, 10)\\
    r_1 &= \frac{1}{\left(\cmu\right)^{-1}\frac{\Omega d^2 \kappa^2}{\sa} + f_{v_2}}
\end{split}
\end{align}
To look more like the original definition of $r$, this can be written as:
\begin{align}
    r &= \min\left(\dfrac{\sa}{\left(\cmu\right)^{-1}\hat{S}\kappa^2d^2}, 10\right)\\
    \hat{S} &= \Omega + \cmu\frac{\sa f_{v_2}}{\kappa^2 d^2}
\end{align}
\subsubsection{Boundary Conditions}
The boundary conditions are:
\begin{align}
    \sa_{wall} &= 0\\
    \sa_{far}  &= 3~ \text{to}~ 5
\end{align}
\end{document}
